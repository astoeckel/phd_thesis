% !TeX spellcheck = en_GB

\newcommand*{\Reals}{\gls{Reals}\xspace}
\newglossaryentry{Reals}{
	name=\ensuremath{\mathbb{R}},
	type=symbols,
	sort=Reals,
	description={The real number-line, i.e., all values of a scalar magnitude},
}

\newcommand*{\Ball}{\gls{Ball}\xspace}
\newglossaryentry{Ball}{
	name=\ensuremath{\mathbb{B}},
	type=symbols,
	sort=Ball,
	description={A hyperball, the set of all $d$ dimensional vectors within radius $r$ of the coordinate origin, in other words
	$r \mathbb{B}^d = \{ \vec x \in \mathbb{R}^d \mid \, \|x\|_2 \leq r \}$,
	where the notation $r \mathbb{B}^d$ denotes the radius $r$ and dimensionality $d$ of the hyperball
	},
}


\newcommand*{\Normal}{\gls{Normal}\xspace}
\newglossaryentry{Normal}{
	name=\ensuremath{\mathcal{N}},
	type=symbols,
	sort=Normal,
	description={The normal distribution. The notation $\nu \sim \mathcal{N}(\mu, \sigma)$ indicates that $\nu$ is a normal-distributed random variable with mean $\mu$ and standard deviation $\sigma$},
}

\newcommand*{\Dec}{\gls{Dec}\xspace}
\newglossaryentry{Dec}{
	name=\ensuremath{\mat{D}},
	type=symbols,
	sort=Dec,
	description={A decoding matrix of shape $\mathbb{R}^{d \times \Npop}$. Decoding matrices are used within the Neural Engineering Framework to either reconstruct the value represented by a population of neurons (the identity decoder) or to decode a specific function. In the latter case, we sometimes write $\mat{D}^f$ to denote the function that is being decoded.},
}

\newcommand*{\Npop}{\gls{Npop}\xspace}
\newglossaryentry{Npop}{
	name=\ensuremath{n},
	type=symbols,
	sort=npop,
	description={Number of neurons in a neuron population.
	This determines the dimensionality of the associated activity vector $\vec a$},
}

\newcommand*{\Nsmpls}{\gls{Nsmpls}\xspace}
\newglossaryentry{Nsmpls}{
	name=\ensuremath{N},
	type=symbols,
	sort=Nsmpls,
	description={This symbol is typically used to indicate the number of samples in an optimization problem},
}

\newcommand*{\Ndim}{\gls{Ndim}\xspace}
\newglossaryentry{Ndim}{
	name=\ensuremath{d},
	type=symbols,
	sort=Ndim,
	description={Number of dimensions represented by a neuron population},
}

\newcommand*{\Xrepr}{\gls{Xrepr}\xspace}
\newglossaryentry{Xrepr}{
	name=\ensuremath{\mathbb{X}},
	type=symbols,
	sort=Xrepre,
	description={Set of possible values represented by a neuron population. We assume that this is a compact set with finite volume, i.e., $\vol(\mathbb{X})$},
}


\newcommand*{\vRest}{\gls{vRest}\xspace}
\newglossaryentry{vRest}{
	name=\ensuremath{v_\mathrm{rest}},
	type=symbols,
	sort=vrest,
	description={Resting potential\index{resting potential} of a neuron.
	This is the membrane potential that can be measured in a neuron when no input is applied.
	In vertebrates the resting potential is typically close to \SI{-65}{\milli\volt}.
	The resting potential is generally equal to the \enquote{leak potential} \EL, though different in semantics}
	% TODO check potentials
}

\newcommand*{\vReset}{\gls{vReset}\xspace}
\newglossaryentry{vReset}{
	name=\ensuremath{v_\mathrm{reset}},
	type=symbols,
	sort=vreset,
	description={Reset potential. TODO}
	% TODO check potentials
}


\newcommand*{\vTh}{\gls{vTh}\xspace}
\newglossaryentry{vTh}{
	name=\ensuremath{v_\mathrm{th}},
	type=symbols,
	sort=vth,
	description={Threshold potential. TODO}
	% TODO check potentials
}


\newcommand*{\CMem}{\gls{CMem}\xspace}
\newglossaryentry{CMem}{
	name=\ensuremath{C_\mathrm{m}},
	type=symbols,
	sort=Cm,
	description={Membrane capacitance. TODO}
}

\newcommand*{\EL}{\gls{EL}\xspace}
\newglossaryentry{EL}{
	name=\ensuremath{E_\mathrm{L}},
	type=symbols,
	sort=EL,
	description={Leak potential. In contrast to the resting potential \vRest, which is an empirical measurement, the leak potential $E_\mathrm{L}$ is the reversal potential of a so-called \enquote{leak-channel}.
	The leak-channel merely a modelling convenience that summarises the state of multiple ion channels of the neuron at rest}
	% TODO check potentials
}

\newcommand*{\gL}{\gls{gL}\xspace}
\newglossaryentry{gL}{
	name=\ensuremath{g_\mathrm{L}},
	type=symbols,
	sort=gL,
	description={Leak conductance. TODO}
}


\newcommand*{\gE}{\gls{gE}\xspace}
\newglossaryentry{gE}{
	name=\ensuremath{g_\mathrm{E}},
	type=symbols,
	sort=gE,
	description={Excitatory conductance. TODO}
}

\newcommand*{\gI}{\gls{gI}\xspace}
\newglossaryentry{gI}{
	name=\ensuremath{g_\mathrm{I}},
	type=symbols,
	sort=gI,
	description={Inhibitory conductance. TODO}
}

