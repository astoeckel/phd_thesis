% !TeX spellcheck = en_GB
%\phantomsection
%\addcontentsline{toc}{chapter}{Acknowledgements}
\begin{center}\textbf{Acknowledgements}\end{center}

Researching and writing a thesis can at times be a rather solitary affair, and a global pandemic surely does little to improve this situation.
More than usual, I am thus grateful to all the wonderful people who have accompanied me throughout this journey.
Foremost, I would like to extend my gratitude to my supervisor Chris Eliasmith.
Chris has been a tremendous source of inspiration and has offered his guidance in countless discussions, while always giving me the freedom to explore what I found most interesting.
Aside from research, Chris went above and beyond to make our lab a fun and rewarding place to be in; our pre-pandemic summer schools and cottage weekends are some of the most memorable times I had during my stay in Canada.

Next, I would like to thank Terry Stewart, who has been an outstanding scientific mentor and source of support throughout the years.
Terry's unparalleled enthusiasm was pivotal in sparking my interest in the Neural Engineering Framework.
It is safe to say that without him, I would not have come to Waterloo and this thesis would not exist.
Similarly, I thank Aaron Voelker for taking the time to read my work on denritic computation and sliding-window transformations, suggesting improvements, and providing instrumental feedback.
Prof.~Javier Medina's (Baylor College of Medicine) CTN seminar served as the inspiration for our cerebellum model.
He furthermore provided helpful feedback on an initial version of the model, which shaped the direction of this work.

All other former members of the CNRG, whom I had many thoughtful discussions with, shall of course not remain unmentioned: Sean Aubin, Peter Blouw, Narsimha Chilkuri, Xuan Choo, Peter Duggins, Nicole Dumont, Michael Furlong, Jan Gosmann, Eric Hunsberger, Pawel Jaworski, Ivana Kajić, Brent Komer, Thomas Lu, Ryan Orr, Sugandha Sharma, Mariah Shein, Sverrir Thorgeirsson, and Natarajan Vaidyanathan.
Unfortunately, due to the pandemic, our traditional board game nights had to come to a sudden end!

Special thanks go to my current and former house mates Brent, Peter, Sean, Xuan, and frequent guest Nicole.
You have bravely endured my unprompted outbursts of talking about my research in the most inappropriate moments, and have in turn pushed me to, at times, actually leave my desk and to have some fun.
If there is any objective measure for what makes marvellous house mates, then it should be \enquote{still talking to each other after being stuck at home for two years}, and you pass this test with flying colours.

On a very fundamental level, this research would not have been possible without thousands of people---most of them volunteers---building the software used for producing this document and for conducting the experiments therein. Just to name a few of those projects: Inkscape, Matplotlib, Numpy, Scipy, \LaTeX, \TeX studio, Poppler, OSQP, CVXOPT, Linux, GNOME, Fedora, GCC, Eigen, Meson, Autograd, Tensorflow, and many more.

Furthermore, I would like to thank all the students that I had the honour of teaching, either as a teaching assistant, or as the instructor of SYDE~556/750 \enquote{Simulating Neurobiological Systems} in Winter 2020.
Thank you for being so curious, and for asking so many fantastic questions!
Even if you have only learned a fraction of what I have learned while teaching you, I would still count this as a great success.
Not only has preparing weekly lecture notes for SYDE~556 made me a more confident writer, but those notes have since become a basis for Chapter~2 of this thesis.

My thesis committee shall of course not remain unmentioned either.
Wulfram Gerstner, Chris Eliasmith, Robin Cohen, Jeff Orchard, and Sue Ann Campbell have all bravely combed through this rather voluminous thesis, provided important feedback on areas that required clarification, and helped to eliminate remaining typos.
Despite these heroic efforts, many mistakes and inconsistencies surely still lurk among the pages to follow.
Those are solely my own responsibility.

On a personal note, I extend my thanks to friends and family---to my brother Daniel Stöckel, and my friend Benjamin Paaßen for their words of wisdom on academia; to Jan and Christina Göpfert for keeping me updated on my former home in Bielefeld; and to Tosca Lechner and Shanna Yang for accompanying me on many walks and hikes.
Last but not least, I owe thanks to my parents Sabine and Thomas Stöckel.
They have taught me to be curious, kindled my interest in technology, supported and encouraged my studies, and, overall, have made me the person I am today.

\bigskip

{
\raggedleft
---\emph{Andreas Stöckel, Waterloo, October 2021}\\
}
