% !TeX spellcheck = en_GB
\begin{center}\textbf{Acknowledgements}\end{center}

Researching and writing a thesis can at times be a rather solitary affair, and a global pandemic surely does little to improve this situation.
More than is perhaps usual, I am thus grateful to all the wonderful people who have accompanied me throughout this journey.

Foremost, I would like to extend my gratitude to my supervisor Chris Eliasmith.
Chris has been a tremendous source of inspiration and has offered his guidance in our countless discussions, while always giving me the freedom to explore what I found most interesting.
Beyond research, Chris has made our lab a fun and rewarding place to be in; our pre-pandemic summer schools and cottage weekends were some of the most memorable experiences I had during my stay in Canada.

Next, I would like to thank Terry Stewart, who has been an outstanding scientific mentor and source of support throughout the years.
Terry's unparalleled enthusiasm has been pivotal in sparking my interest in the Neural Engineering Framework.
It is safe to say that without him, I would not have come to Waterloo and this thesis would not exist.
Similarly, I thank Aaron Voelker for taking the time to read my work on denritic computation and the Legendre Delay Network, suggesting improvements, and providing instrumental feedback.
Prof.~Javier Medina's (Baylor College of Medicine) CTN seminar served as the inspiration for our cerebellum model.
He furthermore provided helpful feedback on an initial version of the model, which shaped the direction of this work.

My thanks further extend to all other former members of the CNRG, all of whom I had many thoughtful discussions with during lab meetings and over lunch: Sean Aubin, Peter Blouw, Narsimha Chilkuri, Xuan Choo, Peter Duggins, Nicole Dumont, Michael Furlong, Jan Gosmann, Eric Hunsberger, Pawel Jaworski, Ivana Kajić, Brent Komer, Thomas Lu, Ryan Orr, Sugandha Sharma, Mariah Shein, Sverrir Thorgeirsson, and Natarajan Vaidyanathan.
Unfortunately, our frequent board game nights had to come to a sudden end!

Of all those people, my particular thanks go to my current and former house mates Brent, Peter, Sean, Xuan, and frequent guest Nicole.
You have bravely endured my unprompted outbursts of talking about research in the most inappropriate moments, and have in turn pushed me to actually leave my desk at times and have some fun.
If there was any objective measure for what makes marvellous house mates, then it might be \enquote{still talking to each other after being stuck at home for almost two years}, and you pass this test with flying colours.

On a very fundamental level, this research would not have been possible without thousands of people---most of them volunteers---building the software used for producing this document and for conducting the experiments therein. Just to name a few of those projects: Inkscape, Matplotlib, Numpy, Scipy, \LaTeX, \TeX studio, Poppler, OSQP, CVXOPT, Linux, GNOME, Fedora, GCC, Eigen, Meson, Autograd, Tensorflow, and many more.

Furthermore, I would like to thank all the students that I had the honour of teaching, either a teaching assistant, or as the instructor of SYDE~556/750 \enquote{Simulating Neurobiological Systems} in Winter 2020.
Thank you for being so curious, and for asking so many fantastic questions!
Even if you had only learned a fraction of what I have learned while teaching you, then this has still been a great success.
Preparing weekly lecture notes for SYDE~556 has made me a more confident writer, and the notes have since become a basis for Chapter~2 in this thesis.

On a personal note, I would like to extend my thanks to friends and family for their emotional support.
To my brother Daniel Stöckel, and my friend Benjamin Paaßen for their words of wisdom on academia; to Jan and Christina Göpfert for keeping me updated on my former home in Bielefeld; and to Tosca Lechner and Shanna Yang for accompanying me on many walks and hikes.
Last but not least, I owe thanks to my parents Sabine and Thomas Stöckel.
They have taught me to be curious, kindled my interest in technology, supported and encouraged my studies, and, overall, have made me the person I am today.

\bigskip

{
\raggedleft
---\emph{Andreas Stöckel, Waterloo, October 2021}\\
}
