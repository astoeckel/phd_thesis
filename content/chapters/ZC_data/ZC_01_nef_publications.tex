% !TeX spellcheck = en_GB

\section{Quantitative Analysis of Publications Related to the NEF}
\label{app:nef_literature}

\begin{figure}[p]
	\centering
	\includegraphics{media/chapters/ZC_data/semantic_scholar_quantitative_data.pdf}%
	{\phantomsubcaption\label{fig:nef_literature_a}}%
	{\phantomsubcaption\label{fig:nef_literature_b}}%
	{\phantomsubcaption\label{fig:nef_literature_c}}%
	\caption[Publications related to the NEF in the S2 ORC dataset]{Publications related to the \NEF in the S2 ORC dataset.
	\textbf{(A)} Citations of key \NEF publications (see text) by affiliation. Affiliation is based on the presence of past and present members of the Computational Neuroscience Research Group (CNRG) at the University of Waterloo in the author list.
	\textbf{(B)} Counts based on the presence of the word \enquote{Nengo} or the phrase \enquote{Neural Engineering Framework} in the title or abstract with manual clean-up for false-positive matches. Split by affiliation as in \emph{(A)}.
	\textbf{(C)} Same data as in \emph{(B)}, but instead categorised according to the topic. Categories are based on keyword-matching in the title with manual clean-up.
	\emph{(*)} Data for 2021 are incomplete.}
	\label{fig:nef_literature}
\end{figure}

We claimed in \Cref{sec:nef} that the Neural Engineering Framework enjoys success as a framework for modelling neurobiological systems, and a technique for programming neuromorphic hardware.
%The purpose of this appendix is to quantify these statements.
Unfortunately, it is challenging to track modelling techniques in the scientific literature.
These methods are often secondary to the subject of a publication and thus likely mentioned only in the main body, and not the freely available title and abstract.

For the purpose of obtaining a rough estimate of the prevalence of the \NEF in the research community, we performed a superficial survey of the scientific literature based on the Semantic Scholar Open Research Corpus (S2 ORC).\footnote{See \url{https://www.semanticscholar.org/}; our analysis is based on the database dated April 1st 2021.}
% 191281586 publications "wc -l"
The S2 ORC is an index of 191 million scientific publications, including title, abstract, authorship, and citation graphs \citep{ammar2018construction}.

As a first query, we track citations of key \NEF publications \citep[specifically][]{eliasmith2003neural,eliasmith2013how,bekolay2014nengo} over time and classify these publications according to the affiliation of the first author.
Results for this are depicted in \Cref{fig:nef_literature_a}.
An increased interest in the \NEF can be observed after 2012; likely due to \citet{eliasmith2012largescale}.
In total, well over 700 external publications cite key work related to the \NEF.
Of course, citations do not reliably indicate the use of the cited technique.%; this number should be treated as an \enquote{upper bound}.

Correspondingly, we also provide a very conservative estimate where we only select publications that either contain the word \enquote{Nengo} or the phrase \enquote{Neural Engineering Framework} in the title or abstract.
After manual clean-up, this results in 138 matching publications, to which we assign a topic (neuromorphics, modelling neurobiological systems, or theory).
%As above, we classify these results by affiliation.
%Additionally, we determine the topic of the publication (neuromorphics, modelling neurobiological systems, or theoretical work).

Results are depicted in  \Cref{fig:nef_literature_b,fig:nef_literature_c}.
Even with this more conservative estimate, about 70 external publications directly mention the \NEF or Nengo in their title or abstract. Roughly half of the publications since 2012 stem from external researchers.
%Topics are, on average, evenly split between publications.
About half of the publications are either related to neuromorphics and neurobiological modelling, whereas the other half falls into the category of theoretical neuroscience, robotics, or software.

Of course, these data are lower bounds; mentions of the \NEF or Nengo in the body of the publication are not captured by this methodology.
%As a further point of reference,
As of writing (May 2021), a Google Scholar search for \texttt{"Neural Engineering Framework"} counts about 700 publications; \texttt{Nengo neural} returns about 640 results.
Unfortunately, there is no official way to access the underlying data.% in a machine-readable format.
