% !TeX spellcheck = en_GB

Human cognition is ultimately grounded in neurophysiological processes.
As suggested by Marr's \enquote{levels of analysis} \citep{marr1976understanding}, cognitive scientists tend to implement models of cognition at algorithmic and computational levels, without explicitly taking limitations of the underlying neural substrate into account \citep{eliasmith2015marr}.

Depending on the hypothesis that is being explored, ignoring biological detail can be reasonable. Yet, a closer look at biology may help in two complementary ways.
First, we can \emph{validate} hypotheses about cognition by determining whether it is possible to implement a particular algorithm under the constraints of the biological network in question.  
Second, we can \emph{generate} new hypotheses by asking what class of algorithms a certain network could support.

We believe that cognitive modelling must ultimately embrace a combination of top-down and bottom-up modelling to narrow down the vast space of possible cognitive science theories and to direct research attention within that space.
However, a central roadblock to the adaptation of such methods is the availability of modelling tools that make it possible to specify detailed biological constraints (e.g., neural response curves, spike rates, synaptic time-constants, connectivity patters) while still being abstract enough to facilitate the specification of high-level cognitive function.

One approach designed to help bridge this gap is the Neural Engineering Framework (NEF; \cite{eliasmith2003neural}), in conjunction with the related Semantic Pointer Architecture (SPA; \cite{eliasmith2013how}).
Up until recently however, it has been unclear how to incorporate certain biological constraints that are often described in the neuroscience literature into NEF networks.
For example, and despite initial progress in this direction \cite{parisien2008solving}, accounting for Dale's principle with purely excitatory and inhibitory neuron populations, as well as incorporating spatial connectivity constraints, has been relatively challenging with the existing NEF-based software tool, Nengo \cite{bekolay2014nengo}.
Furthermore, there have been no studies on how adding these additional constraints influences the high-level function of NEF networks. In this paper, we attempt to address both issues.

Specifically, we describe recent advances in modeling techniques that partially alleviate the shortcomings of the NEF mentioned above.
We then use these methods to construct a model of eyeblink conditioning in the cerebellum, with a focus on the generation of temporal basis function representations in the recurrent Granule-Golgi circuit.

Building a model of temporal representation in the cerebellum is of particular interest, since it remains unclear how exactly the cerebellum manages to learn and reproduce the precise timings observed in eyeblink conditioning. Furthermore, the mechanisms underlying eyeblink conditioning are potentially exploited by cognitive processes as well. Recent evidence---ranging from studies in functional connectivity, neuronal tracing, clinical pathology, to evolutionary physiology---suggests that the tasks supported by the cerebellum are not restricted to motor control alone. The cerebellum may instead be recruited by various brain regions as a \enquote{co-processor} to support brain functions related to higher-order cognition, such as language-based thinking, working memory, perceptual prediction, and tasks requiring precise timings in general \citep{sullivan2010cognitive,
           buckner2013cerebellum,
		   oreilly2008cerebellum,
           e2014metaanalysis}.

Our experiments suggest that---at least under the constraints we consider---the Granule-Golgi circuit is well-suited to encode temporal information using basis functions. This representation can be used in the context of a spiking neural network to learn delays by modulating synaptic weights in the granular-to-Purkinje projection.
Furthermore, we generate hypotheses as to why various biological parameters (such as the sparse connectivity patterns and the time-constants of the neurotransmitters) are as observed.

The remainder of this paper is structured as follows.
We first review the high-level function we hypothesize could be implemented in the Granule-Golgi circuit, followed by an overview of the eyeblink conditioning task, the particular neurophysiological constraints of the cerebellum, and high-level theories of cerebellar function.
We then discuss five neural network implementations with an increasing amount of biological detail, along with the corresponding extensions to the NEF.
We perform a series of experiments that explore the impact of individual parameters on the performance of the increasingly realistic system.
We extend our model to perform the complete eyeblink conditioning task by incorporating the remaining cerebellar microcircuitry and compare the model behavior to empirical data.
Finally, we provide a quick overview of \enquote{NengoBio,} the open-source addon to Nengo we developed to encode the aforementioned biological constraints.\footnote{See \url{https://github.com/astoeckel/nengo-bio}.}
