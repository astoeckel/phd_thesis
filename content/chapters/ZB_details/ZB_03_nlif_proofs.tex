% !TeX spellcheck = en_GB

\section{Proofs for Chapter~3}
\label{app:neural_network_proofs}

In this section, we provide proofs for the claims made in \Cref{chp:nlif}, where we discussed theoretical aspects of neural networks in general, and dendritic computation in particular.

\subsection{Proof of Theorem~3.1}

In \Cref{sec:additive_net} we claimed that there are always functions $f$ that cannot be approximated using an additive network with a fixed nonlinearity $\sigma$.
We prove this by constructing an $f$ that is constant along the sides of a rectangle, but maps onto a different value on the inside.

\ThmXorGeneral*

\begin{proof}
Since $\mathbb{X}$ is a compact set with dimensionality $\dim(\mathbb{X}) = \ell$, there must be a rectangular subset of the following form
$[a_0, a_1] \times [b_0, b_1] \times \{c_1\} \times \ldots \times \{c_{\ell - 2}\} \subseteq \mathbb{X}$.
Now, consider an $f$ with
\begin{align*}
				     &f(a_0, x_2, c_1, \ldots, c_{\ell - 2}) = d \quad \text{for } x_1 \in [a_0, a_1] \,, \\
	\text{and} \quad &f(x_1, b_0, c_1, \ldots, c_{\ell - 2}) = d \quad \text{for } x_2 \in [b_0, b_1] \,, \\
	\text{and} \quad &f(x_1, x_2, c_1, \ldots, c_{\ell - 2}) \neq d \quad \text{for } x_1 \in (a_0, a_1] \text{ and } x_2 \in (b_0, b_1] \,.
\end{align*}
Since $\sigma$ is continuous, the functions $f_1(x_1)$, $f_2(x_2)$ must be constant for all $x_1 \in [a_0, a_1]$ and $x_2 \in [b_0, b_1]$, otherwise the first two conditions cannot be met.
However, with $f_1$ and $f_2$ being constant over the indicated intervals, the second condition cannot be met. Hence, $f$ is a function that cannot be computed using additive networks, independent of $\sigma$.
\end{proof}

\subsection{Proof of Theorem~3.2}
\label{app:thm_weak_xor}

\Citet[Chapter~2]{minsky1987perceptrons} note that the Perceptron cannot compute \XOR.
This notion can be extended to the more general \enquote{additive networks} discussed in \Cref{sec:dendritic_computation_theory}, as well as a weaker notion of the \XOR problem (\Cref{def:weak_xor}).
In particular, we claimed the following:

\ThmWeakXor*

\begin{proof}
Suppose $\phi(x_1, x_2)$ could solve the weak \XOR problem. Substitute $a_0' = f_1(a_0)$, $a_1' = f_1(a_1)$, $b_0' = f_2(b_0)$, $b_1' = f_2(b_1)$, where $a_0$, $a_1$, $b_0$, $b_1$ are from \Cref{def:weak_xor}.
Further expanding this definition by substituting in our additive network $\phi(x_1, x_2)$ we obtain
	\begin{align*}
	&\hphantom{\wedge} \;    \left( \sigma(a_0' + b_0') < \sigma(a_0' + b_1') \right)
	 \wedge   \left( \sigma(a_1' + b_1') < \sigma(a_1' + b_0') \right) \\
	&\wedge   \left( \sigma(a_0' + b_0') < \sigma(a_1' + b_0') \right) 
	 \wedge   \left( \sigma(a_1' + b_1') < \sigma(a_0' + b_1') \right) \,.
	\end{align*}
	Assume without loss of generality that $\sigma$ is monotonically increasing. Then, the first line implies $\big(b_0' < b_1' \big) \wedge \big( b_0' > b_1' \big)$ and the second line $\big(a_0' < a_1' \big) \wedge \big( a_0' > a_1' \big)$. \Lightning
\end{proof}

\subsection{Proof of Theorem~3.3}
\label{app:thm_multiplication}

Additive networks are capable of solving the weak \XOR problem if we allow a non-monotonic neural nonlinearity $\sigma$.
However, as we already saw in \Cref{thm:xor_general}, being able to freely choose $\sigma$ does not mean that we can universally approximate any function.
An example we gave in \Cref{thm:multiplication} is that of multiplication.
While this is merely a special case of the kind of function used to prove \Cref{thm:xor_general}, we provide an alternative proof below.

\ThmMultiplication*

\begin{proof}
Assume that continuous $f_1$, $f_2$, $\sigma$ exist such that $\sigma(f_1(x_1) + f_2(x_2)) = x_1 x_2$.
Let $\xi_\mathrm{min} \neq \xi_\mathrm{max}$ be the extrema of $f_1$ over $[0, 1]$.
These extrema exist because $f_1$ is continuous on a compact set; hence, by the Heine-Cantor theorem, $f_1$ is uniformly continuous and not unbounded.


\begin{figure}[t]
	\centering
	\includegraphics{media/chapters/ZB_details/multiplication_theorem.pdf}
	\caption[Visualisation of the proof of Theorem~\ref{thm:multiplication}]{Visualisation of the proof of Theorem~\ref{thm:multiplication}.
	We can find a $\delta$ such that the intervals spanned by $f_1(x_1) + f_2(0)$ and $f_1(x_2) + f_2(\delta)$ overlap (hatched region).
	This should not be possible, given that $\sigma(f(x_1) + f(0)) = 0$ and $\sigma(f_1(x_1) + f_2(\delta)) \neq 0$.
	Coloured background corresponds to the desired product $x_1 x_2 = \sigma(f_1(x_1) + f_2(x_2))$.
	This colour should be the same along the horizontals of this diagram.}
	\label{fig:multiplication_theorem}
\end{figure}

Consider $x_2 = 0$ and $x_1 \in [0, 1]$.
The expression $f_1(x_1) + f_2(0)$ covers the interval $[\xi_\mathrm{min} + f_2(0) , \xi_\mathrm{max} + f_2(0)]$.
Since $f_2$ is continuous, there must be, according to the epsilon-delta definition of continuity, a $\delta \neq 0$ such that $f_2(\delta) - f_2(0) < \xi_\mathrm{max} - \xi_\mathrm{min}$.
Hence, as illustrated in \Cref{fig:multiplication_theorem}, the following two sets are not disjoint
\begin{align*}
	\bigl\{ f_1(x_1) + f_2(0) \mid x_1 \in (0, 1] \bigr\} \cap \bigl\{ f_1(x_1) + f_2(\delta) \mid x_1 \in (0, 1] \bigr\} \neq \emptyset \,.
\end{align*}
But, according to our assumption, it should also hold that
\begin{align*}
	\sigma\bigl(f_1(x_1) + f_2(0)\bigr) &= x_1 \cdot 0 = 0 \text{ for all } x_1 \in (0, 1] \\
	\text{and} \quad \sigma\bigl(f_1(x_1) + f_2(\delta)\bigr) &= x_1 \cdot \delta \neq 0 \text{ for all } x_1 \in (0, 1] \,.
\end{align*}
However, since the values passed to $\sigma$ overlap in both cases, this is a contradiction. \Lightning
\end{proof}

\subsection{Discussion of Theorem~3.4}
\label{app:thm_two_layer_universal}

\begin{figure}
	\centering
	\includegraphics{media/chapters/ZB_details/universality_proof.pdf}%
	{\phantomsubcaption\label{fig:universality_proof_a}}%
	{\phantomsubcaption\label{fig:universality_proof_b}}%
	\caption[Illustration of \Cref{thm:two_layer_universal}]{Illustration of \Cref{thm:two_layer_universal}. \textbf{(A)} 1D ensemble with step activation function. Lines are tuning curves, black arrows the encoder. For $n \to \infty$ neurons, every point on the $x$-axis is enclosed by two neighbouring tuning-curves. Subtracting and scaling these two tuning curves appropriately, we can assign a value $f(x)$ to all $x \in [0, 1]$.
	For example, the two highlighted tuning curves, enclose a small region on the $x$-axis (blue highlight).
	\textbf{(B)} 2D ensemble with step activation. Lines are the intercept of an individual neuron, arrows the encoder.
	The tuning curves can be linearly combined such that any convex region in the space is assigned a relatively large value.
	The depicted weighting encloses the region marked by the cross. Blue corresponds to positive, red to negative decoded values. The highlighted intercepts correspond to the neurons with the largest weights.}
\end{figure}

\Cref{thm:two_layer_universal} claims that a single \NEF ensemble, i.e., a multi-layer network with a single hidden layer, is a universal function approximator.
More precisely, we stated the following:

\ThmTwoLayerUniversal*

This universality of two-layer networks of this form is a well known fact, and has, in a more general form, first been proved by \citet{hornik1989multilayer}.
In the terminology of the authors, the above equation describes a $\Sigma$-network, and $\sigma$ is a \enquote{squashing function}.
A notable difference to our theorem is that Hornik et al., define $\alpha_i \langle \vec e_i, \vec x \rangle + \beta_i$ to be all possible affine functions mapping from $\mathbb{R}^\ell$ onto $\mathbb{R}$, while we limit the affine functions to those required to cover the unit-ball $\mathbb{B}^\ell$.

Instead of providing a complete proof, we would like to suggest geometrically why this theorem holds.
Consider the case where $\sigma$ is a step-function, i.e., $\sigma(\xi) = 1$ if $\xi > 0$, otherwise $\sigma(\xi) = 0$.
In the case of a one-dimensional network, and as is depicted in \Cref{fig:universality_proof_a}, it is apparent that as the number of neurons $n$ goes to infinity, we can enclose any point on the $x$-axis with arbitrary precision by subtracting two appropriate neuron tuning curves with equal encoder.
By scaling the two tuning curves, we can hence assign a value $f(x)$ to each represented value $x$.

Similarly, for higher-dimensional representations, we can linearly combine tuning curves such that any convex region enclosed by the neural intercepts is assigned an arbitrary value $f(\vec x)$, while all other regions are assigned values close to zero (cf.~\Cref{fig:universality_proof_b}).
As we increase the number of neurons, and if the conditions stated in the theorem are met, the enclosed regions converge to individual points.
We can approximate any function $f(\vec x)$ by scaling and summing the individual decodings.

\subsection{Proof of Theorem~3.6}
\label{app:thm_nlif_convergence}

Back in \Cref{sec:nlif_subth_properties} we claimed that the feedback matrix of the \nlif dynamical system is stable and does not possess any oscillatory behaviour.
To prove this, we first show the following lemma concerning the Laplacian quadratic form (cf.~\cite[Section~18.3.2]{spielman2012spectral}):
\begin{lemma}
\label{lem:laplacian_quadratic}
Let $\mat L$ be the Laplacian of a positively weighted connected graph with weights $c_{ij}$ and edges $E$.
Then $\vec x = \alpha \vec{1}$ with $\alpha \in \mathbb{R}$ is the only root (besides $\vec x = \vec{0}$) of the Laplacian quadratic form
\begin{align*}
	\vec x^T \mat L \vec x = \sum_{(i, j) \in E} c_{ij} (x_i - x_j)^2 \,.
\end{align*}
\end{lemma}
\begin{proof}
Because of all $c_{ij}$ being positive, the quadratic form is clearly non-negative. In other words, $\mat L$ is positive semidefinite and all its eigenvalues are nonnegative.
Since $\mat L$ is symmetric, its eigen-decomposition is $\mat Q \mat \Lambda \mat Q^T$, where $\mat Q$ is orthogonal.
Let $\vec y = \mat Q^T \vec x$.
It holds
\begin{align*}
	\vec x^T \mat L \vec x = \vec y^T \mat \Lambda \vec y = \sum_{i = 1}^n y_i^2 \lambda_i = 0 \Leftrightarrow \mat \Lambda \vec y = \vec 0 \text{ since } \lambda_i \geq 0\,.
\end{align*}
Since $\mat Q$ is orthogonal, it must be full-rank.
This implies that its null-space only contains the zero-vector.
Correspondingly, $\vec x^T \mat Q \mat \Lambda \mat Q^T \vec x = 0$ exactly if $\mat Q \mat \Lambda \mat Q^T \vec x = \mat L \vec x = \vec 0$, or, put differently, $\vec x$ is in the null-space of $\mat L$.

By Theorem~13.1.1 in \citet[Chapter~13]{godsil2001algebraic}, the multiplicity of the eigenvalue zero in the Laplacian $\mat L$ is one, with eigenvector $\vec 1$.
Hence, the null-space of $\mat L$ is $\alpha \vec 1$.
\end{proof}

Equipped with this Lemma, we can now show the original theorem.

\ThmNlifConvergence*

\begin{proof}
The feedback matrix $\mat A[\vec g] \diag(\vCm)^{-1}$ has the following form:
\begin{align*}
	\mat A[\vec g] \diag(\vCm)^{-1}
	&= -\bigl( \mat L + \diag(\vec a' + \mat A' \vec g) \bigr) \diag(\vCm)^{-1} 
	 = -\bigl( \mat L + \diag(\vec \epsilon) \bigr) \diag(\vCm)^{-1}  \,,
\end{align*}
where $\mat L$ is the graph Laplacian, $\vCm$ is a positive vector of membrane capacitances, and $\vec \epsilon$ is, by the conditions specified in the theorem (i.e., there being at least one conductance-based channel), a non-negative vector with $\vec \epsilon \neq \vec 0$.
Eliminating the minus sign, the following must hold to ensure negative definiteness:
\begin{align*}
	 \vec x^T \mat L' \vec x 
	 	= \vec x^T \bigl( \mat L + \diag(\vec \epsilon) \bigr) \diag(\vec \vCm)^{-1} \vec x
	 	> 0 \quad\quad \text{for all} \quad \vec x \neq 0 \,.
\end{align*}
Note that multiplication with $\diag(\vCm)^{-1}$ (i.e., scaling each row of the Laplacian by the inverse of the corresponding \Cm) turns the undirected graph into a directed graph.
However, we can still expand this expression using the Laplacian quadratic form \citep[cf.][Section~18.3.5]{spielman2012spectral}.
Again, let $E$ be the set of edges and $c_{ij}$ the connection weights.
We get
\begin{align*}
	\vec x^T \mat L' \vec x
	&= \vec x^T \bigl( \mat L + \diag(\vec \epsilon) \bigr) \diag(\vCm)^{-1} \vec x
	 = \vec x^T \mat L \diag(\vCm)^{-1} \vec x
	 + \vec x^T \diag(\vec \epsilon) \diag(\vCm)^{-1} \vec x \\
	&= \sum_{(i, j) \in E} c_{ij} \CmiInv (x_i - x_j)^2 + \sum_{i = 1}^n \epsilon_i \CmiInv x_i^2 \,.
\end{align*}
This expression is obviously non-negative due to $c_{ij}$, \CmiInv being positive and $\epsilon_i$ being non-negative.
However, to show strict positivity, and correspondingly, negative definiteness of our original expression, consider the conditions under which $\vec x^T \mat L' \vec x$ could be zero:
\begin{enumerate}[(i)]
	\item The sum over $c_{ij} \CmiInv (x_i - x_j)^2$ is zero exactly if there are no edges in the graph (since the graph is connected, this is only possible for $n = 1$), or, under the conditions listed in \Cref{lem:laplacian_quadratic}, that is $\vec x = \alpha \vec{1}$, or, put differently, $x_i = x_j$ for all $i$, $j$.
	\item The sum over $\epsilon_i \CmiInv x_i^2$ is zero exactly if $\epsilon_i \neq 0$ implies that $x_i = 0$.
\end{enumerate}

Crucially, \Cref{lem:laplacian_quadratic} still applies in \emph{(i)} despite the multiplication with $\CmiInv$.
This is because $\CmiInv$ and $c_{ij}$ are strictly positive, and we know that the only way for $c_{ij} (x_i - x_j)^2$ to evaluate to zero is that all summands evaluate to zero.

To complete the proof, first consider $n = 1$, as suggested by \emph{(i)}.
In this case, since $\vec x \neq \vec{0}$ it follows $x_1 \neq 0$. Furthermore $\epsilon_1 > 0$.
Hence, $\epsilon_i x_i^2 > 0$ and $\vec x \mat L' \vec x^T$ is strictly positive.

Now assume that $n > 1$. Since $\vec x \neq \vec{0}$, and at least one $\epsilon_i$ is non-zero, this, by \emph{(ii)}, implies that the corresponding $x_i = 0$.
By \emph{(i)}, $\vec x \mat L' \vec x^T$ can only evaluate to zero if $\vec x = \vec{0}$, which we excluded.
Hence, $\vec x \mat L' \vec x^T > 0$ for all $\vec x \neq 0$ and $\mat L'$ is positive definite.
This implies that all eigenvalues of $\mat L'$ are non-zero and that $\mat L'$ is non-singular.
Our original claim follows trivially.
\end{proof}

%To prove this, we will make use of the following lemmas, that, in similar form, can be found in the literature on graph theory; see the provided references for more information.
%
%\begin{lemma}
%The Laplacian $\mat L$ of a positively weighted connected graph is positive semi-definite---all eigenvalues are real and non-negative.
%The smallest eigenvalue is zero with multiplicity one.
%\end{lemma}
%
%\begin{proof}
%Multiplicity of the eigenvalue zero follows from \citet[Chapter~13, Theorem~13.1.1]{godsil2001algebraic}.
%There is one connected component in the graph, hence the rank of the Laplacian $\mat L$ is $n - 1$ and there must be exactly one eigenvalue zero.
%Positive semi-definiteness directly follows from the quadratic form of the Laplacian \citep[Section~18.3.2]{spielman2012spectral}:
%\begin{align*}
%	\vec x \mat L \vec x^T &= \sum_{(i, j) \in E} c_{ij} (x_i - x_j)^2 \geq 0 \,,
%\end{align*}
%where $\vec x \in \mathbb{R}^n$, $E$ are the edges in the graph, and $c_{ij}$ are the connection weights; nonnegativity holds because of $c_{ij}$ being nonnegative.
%This is the definition of \enquote{positive semidefinite}.
%Since $\mat L$ is symmetric, \enquote{positive semidefinite} is equivalent to \enquote{all eigenvalues are non-negative}.
%\end{proof}
%
%\begin{lemma}
%Let $\mat L$ the Laplacian of a positively weighted connected graph. Then $\mat L' = \mat L + \diag(\vec{\epsilon})$ is non-singular for any non-negative $\vec{\epsilon} \in \mathbb{R}^n \setminus \{\vec{0}\}$.
%\end{lemma}

%\begin{proof}
%Non-singularity is equivalent to the null-space of $\mat L'$ only containing the zero-vector.
%Assume that $\vec x \neq \vec{0}$ such that $(\mat L' \vec x)_i = 0$ for all $i$.
%By definition of the graph Laplacian we have
%\begin{align*}
%	\bigl( \mat L' \vec x \bigr)_i &= \sum_{j = 1}^n c_{ij} x_i - \sum_{j = 1}^n c_{ij} x_j + \epsilon_i x_i = 0 \quad \text{for all } i \,.
%\end{align*}
%Correspondingly, the sum over all $i$ should be zero as well.
%Exploiting the symmetry of the undirected graph (i.e., $c_{ij} = c_{ji}$) we obtain
%\begin{align*}
%	\sum_{i = 1}^n \bigl( \mat L' \vec x \bigr)_i
%		&= \sum_{i = 1}^n \left( \sum_{j = 1}^n c_{ij} x_i - \sum_{j = 1}^n c_{ij} x_j + \epsilon_i x_i \right)
%		 = \left( \sum_{i = 1}^n \sum_{j = 1}^n c_{ij} x_i \right)
%		 - \left( \sum_{i = 1}^n \sum_{j = 1}^n c_{ji} x_j \right)
%		 + \sum_{i = 1}^n \epsilon_i x_i \\
%		&= \left( \sum_{i = 1}^n \sum_{j = 1}^n c_{ij} x_i \right)
%		 - \left( \sum_{j = 1}^n \sum_{i = 1}^n c_{ij} x_i \right)
%		 + \sum_{i = 1}^n \epsilon_i x_i
%		 = \langle \vec x, \vec \epsilon \rangle
%		 = 0 \,.
%\end{align*}
%This equality can only hold if $\vec x$ is orthogonal to $\vec{\epsilon}$, i.e., $\langle \vec x, \vec \epsilon \rangle = 0$.
%\end{proof}

\subsection{Proof of Theorem~3.7}
\label{app:nlif_product_terms}

We stated in \Cref{sec:nlif_examples} that the dendritic nonlinearity $H$ of an $n$-LIF neuron only contains product terms between the input channels of different compartments, but not between input channels of the same compartment.
In particular, we claimed the following:

\ThmNlifProductTerms*

\begin{proof}
We first prove claim \emph{(A)}, followed by both \emph{(B)} and \emph{(C)}.
However, to do so, we first need to lay out the problem at hand in more detail.
First, note that all $\ell$ branches of the neuron, including the soma itself, form block matrices $\mat{\tilde C}_m[\vec g]$ within the reduced system matrix $\mat{\tilde A}[\vec g]$. 
In other words, the reduced system matrix and its inverse are given as
\begin{align}
\mat{\tilde A}[\vec g] =
\NiceMatrixOptions{renew-dots,renew-matrix}
\begin{bmatrix}
	1	   & 0                        & \cdots & 0 \\
	0      & \mat{\tilde C}_1[\vec g] & \ddots & \vdots \\
	\vdots & \ddots                   & \ddots & 0 \\
    0      & \cdots                   & 0      & \mat{\tilde C}_\ell[\vec g]
\end{bmatrix}
\quad\quad
&\Leftrightarrow
\quad\quad
\mat{\tilde A}[\vec g]^{-1} =
\NiceMatrixOptions{renew-dots,renew-matrix}
\begin{bmatrix}
	1	   & 0                             & \cdots & 0 \\
	0      & \mat{\tilde C}_1[\vec g]^{-1} & \ddots & \vdots \\
	\vdots & \ddots                        & \ddots & 0 \\
    0      & \cdots                        & 0      & \mat{\tilde C}_\ell[\vec g]^{-1}
\end{bmatrix} \,,
\label{eqn:h_model_blocks}
\end{align}
To simplify writing out the block matrix $\mat{\tilde C}_\ell[\vec g]$, let $r = i_{m - 1} + 1$ be the index of the first compartment in each branch, and $s = i_m$  be the index of the last compatment in each branch. Each block matrix is of the form
\begin{align*}
\mat{\tilde C}_m[\vec g] =
\NiceMatrixOptions{renew-dots,renew-matrix}
\begin{bmatrix}
	 d_{r}[\vec g] & -c_{r, r + 1} & \cdots            & -c_{r, s} \\
	-c_{r + 1, r}  & \ddots        & \ddots            & \vdots \\
	\vdots         & \ddots        & \ddots            & -c_{s - 1, r} \\
    -c_{s, r}      & \cdots        & -c_{s, r - 1}     & d_{s}[\vec g]
\end{bmatrix} \,,
\end{align*}
where, by construction of $\mat{\tilde A}[\vec g]$, each diagonal element $d_i[\vec g]$ is an affine function in the conductance-based input channels.
At the same time, each entry in the input vector $(\vec{\tilde b}[\vec g])_i$ is an affine function in both the conductance- and current-based input channels targeting this channel.
We have
\begin{align*}
	d_i[\vec g] &= \tilde a_0^i + \sum_{j = 1}^{k} \tilde a_j^i g_j^i \,, & 
	(\vec{\tilde b}[\vec g])_i &= \tilde b_0^i + \sum_{j = 1}^{k} \tilde b_j^i g_j^i + \sum_{k = 1}^{k} \tilde c_j^i J_j^i \,,
\end{align*}
where $\tilde a^i_j$, $\tilde b^i_j$, $\tilde c^i_j$ are the corresponding system matrix entries.
Finally, recall \cref{eqn:h_model}, i.e., the mapping between the input $\vec g$ and the predicted somatic current:
\begin{align}
	H(\vec g) &\approx \sum_{i = 1}^n \tilde c_i (\tilde v^\mathrm{eq}_i - \vSom) \,,
	\quad\quad \text{where} \quad \vec{\tilde v}^\mathrm{eq} = -{\mat{\tilde A}}[\vec g]^{-1} \vec{\tilde b}[\vec g] \,.
	\label{eqn:h_model_app}
\end{align}

\paragraph{Claim (A)}
The first claim (i.e., the structure of the affine function $H_0$) trivially follows from multiplying the inverse of the first \enquote{somatic block} of the reduced system matrix $\mat{\tilde A}[\vec g]$ (i.e., the identity) with the first entry of the reduced input vector $(\vec{\tilde b}[\vec g])_1$.
This results in first entry of $\vec{\tilde v}^\mathrm{eq}$; further combining this with \cref{eqn:h_model_app} yields an affine function $H_0(g_1^1, \ldots, g_k^1, J_1^1, \ldots, J_k^1)$.

\paragraph{Claims (B) and (C)}
The last two claims, i.e., the somatic current model $H$ being a sum of rational functions with the given structural constraints, can be obtained by systematically inverting the individual block matrices $\mat{\tilde C}_m[\vec g]$.

In particular, note that the sum-of-rational-functions structure directly follows from plugging $\mat{\tilde A}[\vec g]^{-1}$ from \cref{eqn:h_model_blocks} into \cref{eqn:h_model_app}.
As we will become apparent in a moment, the product between the inverted block matrix $\mat{\tilde C}_m[\vec g]$ for branch $m$ and the corresponding section of the input vector $\vec{\tilde b}[\vec g]$ form the individual rational functions.
By construction, this rational function can only depend on the input channels targeting the branch $m$.
That is, each sum term $m$ can only depend on the variables $g_1^{r}, \ldots, g_k^{s}, J_1^{r}, \ldots, J_k^{s}$, where $r$, $s$ are as defined above.

To see that the product between the inverted block matrix and the input vector forms a rational function, remember that the inverse of a matrix can be written as its adjoint scaled by the inverse of its determinant \citep[e.g.,][Theorem~1.9, p.~366]{hefferon2020linear}:
\begin{align*}
	\mat{\tilde C}_m[\vec g]^{-1} &= \frac{1}{\det(\mat{\tilde C}_m[\vec g])} \adj(\mat{\tilde C}_m[\vec g]) \,.
\end{align*}
Crucially, the determinant fully determines the denominator $H^A_m$ of each rational function.
The division by the determinant is the only division in each product between the inverted block-matrix $\mat{\tilde C}_m[\vec g]^{-1}$ and the corresponding portion of the input vector $\vec{\tilde b}[\vec g]$.

In addition to the properties listed above, claim \emph{(B)} states that $H^A_m$ only depends on the conductance-based inputs, and that it only contains products between inputs targeting different compartments.
The first statement follows from $\mat{\tilde A}[\vec g]$ only containing conductance-based inputs by construction.
The second statement follows from the permutation expansion of the determinant \citep[e.g.,][Section~4.I.3, p.~337]{hefferon2020linear}, we have
\begin{align*}
	H^A_m(g_1^I, \ldots, g_k^J) = \det( \mat{\tilde C}_m[\vec g] )
		&= \sum_{\pi \in \mathbb{P}} \sign(\pi) \prod_{i=1}^n \bigl(\mat{\tilde C}_m[\vec g]\bigr)_{i, \pi(i) + 1} \,,
	& \text{where } \mathbb{P} = S({1, \ldots, n - 1}) \,.
\end{align*}
Here, $S(X)$ denotes the set of permutations of the set $X$, and $\sign(\pi) \in \{-1, 1\}$ is the \enquote{signum} of the permutation $\pi$.
Importantly, each product in the determinant only contains each diagonal element $d_i[\vec g]$ exactly once and there are no product-terms between inputs targeting the same compartment $i$.

Finally, claim \emph{(C)} states that the numerators $H_m^B$ only contains product terms between inputs targeting different compartments, and that each product term features at most one current-based input.
The second statement is a result of $\mat{\tilde A}[\vec g]$ only depending on conductance-based inputs, and all current-based inputs only influencing $\vec{\tilde b}[\vec g]$.
Multiplications with current-based inputs are a result of the final matrix-vector product between $\mat{\tilde A}^{-1}[\vec g]^{-1}$ and the input matrix $\vec{\tilde b}[\vec g]$ and there can be only one current-based input per product term.

The first statement follows from the structure of adjoint.
The numerator $H_m^B$ is determined by the matrix product between the adjoint of $\mat{\tilde C}_m[\vec g]$ and the corresponding portion of $\vec{\tilde b}[\vec g]$.
Notably, the adjoint of a matrix $\mat A$ is a matrix of determinants of $\mat A$ where the $k$th row and the $\ell$th column have been deleted:
\begin{align*}
	\bigl(\adj(\mat{\tilde C}_m[\vec g])\bigr)_{k\ell}
		&= \sum_{\pi \in \mathbb{P}} \sign(\pi) \prod_{\substack{i=1\\i \neq k}}^n \bigl( \mat{\tilde C}_m[\vec g] \bigr)_{i, \pi(i) + 1} \,,
	& \text{where } \mathbb{P} = S({1, \ldots, \ell - 1, \ell + 1, n - 1}) \,.
\end{align*}
Hence, the entry $(\adj(\mat{\tilde C}[\vec g]))_{k\ell}$ neither contains the term $d_{k}[\vec g]$, nor $d_{\ell}[\vec g]$, and the product-terms formed by computing the matrix-vector product between $\adj(\mat{\tilde C}_m[\vec g])$ and $\vec{\tilde b}[\vec g]$ contain each input channel at most once.
This concludes our proof.
\end{proof}

\subsection{Proof of Theorem~3.8}
\label{app:thm_two_comp_xor}

We noted in \Cref{sec:nlif_examples} that the two-compartment \LIF nonlinearity cannot be used to solve the weak \XOR problem.
More precisely, we stated the following:

\ThmTwoCompXor*

The proof for this is analogous to the proof that additive networks cannot solve the \XOR problem (cf.~\Cref{app:thm_weak_xor}).
Particularly, we rely on the denominator being strictly positive.

\begin{proof}
For $b_0 \neq 0$, $H$ as given in the above theorem can be reparametrised to $H'$ as follows
\begin{align*} 
	H(g_\mathrm{E}, g_\mathrm{I}) &= H'\left( \dfrac{b_1  g_\mathrm{E}}{| b_0 |}, \dfrac{b_2 g_\mathrm{I}}{| b_0 |} \right) = H'(x, y) = \frac{\pm 1 + x - y}{c_0 + c_1 x + c_2 y} \,,
\end{align*}
where $c_0 > 0 \text{ and } c_1, c_2, x, y \geq 0$.
Assume that $\phi(x, y) = H'(x, y)$ can solve the weak \XOR problem. Since the denominator in the above nonlinearity is strictly positive, we can safely cross-multiply with the denominator across the inequalities and apply the above definition
\begin{align*}
		&\hphantom{\wedge}\;\,  \bigl( 0 < \hphantom{-}  x_0 y_0 c_1 + x_0 y_0 c_2 - x_0 y_1 c_1 - x_0 y_1 c_2 + y_0 c_0 \pm y_0 c_2 - y_1 c_0 \mp y_1 c_2 \bigr) \\
		&\wedge  \bigl( 0 <           -   x_1 y_0 c_1 - x_1 y_0 c_2 + x_1 y_1 c_1 + x_1 y_1 c_2 - y_0 c_0 \mp y_0 c_2 + y_1 c_0 \pm y_1 c_2 \bigr) \\
		&\wedge  \bigl( 0 <           -   x_0 y_0 c_1 - x_0 y_0 c_2 + x_1 y_0 c_1 + x_1 y_0 c_2 - x_0 c_0 \pm x_0 c_1 + x_1 c_0 \mp x_1 c_1 \bigr) \\
		&\wedge  \bigl( 0 < \hphantom{-}  x_0 y_1 c_1 + x_0 y_1 c_2 - x_1 y_1 c_1 - x_1 y_1 c_2 + x_0 c_0 \mp x_0 c_1 - x_1 c_0 \pm x_1 c_1 \bigr) \,.
\end{align*}
This can be simplified to
\begin{align*}
		&\hphantom{\wedge}\;\, \bigl( 0 < \hphantom{-} ((c_1 + c_2) x_0 \pm c_2 + c_0) (y_0 - y_1) \bigr)
		 \wedge  \bigl( 0 <           -  ((c_1 + c_2) x_1 \pm c_2 + c_0) (y_0 - y_1) \bigr) \\
		&\wedge  \bigl( 0 <           -  ((c_1 + c_2) y_0 \mp c_1 + c_0) (x_0 - x_1) \bigr)
		 \wedge  \bigl( 0 < \hphantom{-} ((c_1 + c_2) y_1 \mp c_1 + c_0) (x_0 - x_1) \bigr) \,.
\end{align*}
Due to  the nonnegativity constraints either the first line implies $(y_0 - y_1 > 0) \wedge (y_0 - y_1 < 0)$ (for the \enquote{$+$} branch of the \enquote{$\pm$}), or the second line implies $(x_0 - x_1 < 0) \wedge (x_0 - x_1 > 0)$ (for the \enquote{$+$} branch of the \enquote{$\mp$}), which is a contradiction. The argument for $b_0 = 0$ is similar. Thus, the theorem holds. In contrast to the previous proof no contradiction can be derived for both lines at the same time. In other words, there are valid parameters $c_0$, $c_1$, $c_2$ for which there exist $x_0$, $y_0$, $x_1$, $y_1$ such that two of the four inequalities hold.
\end{proof}
