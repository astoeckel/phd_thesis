% !TeX spellcheck = en_GB

\section{An $n$-LIF Weight Solver and Simulator: libnlif}
\label{app:libnlif}

\emph{libnlif} is a hybrid Python and C++ library that facilitates working with \nlif neurons, specifically implementing the techniques discussed in \Cref{sec:nlif} onward.
This includes providing a simple \API for describing and simulating \nlif neurons, as well as an implementation of the various synaptic weight and parameter optimisers from \Cref{sec:two_comp_lif,sec:nlif_opt}.

\subsection{Describing $n$-LIF Neurons}

The user-facing \API for describing neuron models is heavily inspired by Nengo.
For example, a standard \LIF neuron with current-based input (cf.~\Cref{fig:nlif_a}) can be constructed as follows:
\begin{pythoncode}
import nlif

with nlif.Neuron() as lif: # Create a new neuron description 
	with nlif.Soma(v_th=-50e-3, tau_ref=2e-3, tau_spike=1e-3, C_m=1e-9) as soma:
		gL = nlif.CondChan(g=50e-9, E_rev=-80e-3) # Static leak channel
		J = nlif.CurChan(mul=1.0)                 # Current-based input channel

lif_assm = two_comp_lif.assemble() # Assemble an immutable representation
\end{pythoncode}
Similarly, the following code describes the two-compartment \LIF neuron (cf.~\Cref{fig:nlif_c}):
\begin{pythoncode}
with nlif.Neuron() as two_comp_lif:
	with nlif.Soma(v_th=-50e-3, tau_ref=2e-3, tau_spike=1e-3, C_m=1e-9) as soma:
		gL = nlif.CondChan(g=50e-9, E_rev=-65e-3)  # Static leak channel
	with nlif.Compartment(C_m=1e-9) as dendrites:
		gL = nlif.CondChan(g=50e-9, E_rev=-65e-3)  # Static leak channel
		gE = nlif.CondChan(E_rev=0e-3)             # Excitatory input channel
		gI = nlif.CondChan(E_rev=-75e-3)           # Inhibitory input channel
	nlif.Connection(soma, dendrites, g_c=50e-9)
two_comp_lif_assm = two_comp_lif.assemble()
\end{pythoncode}
This overall pattern extends to arbitrary connectivity graphs.
Individual objects and the graph structure are validated for adherence to the \nlif constraints (\Cref{sec:nlif_description}) in the \texttt{assemble} method.
The Python \enquote{with} statement is used to establish an object hierarchy; this is accomplished by overriding the \texttt{\_\_enter\_\_} and \texttt{\_\_exit\_\_} functions and tracking the current parent object in a thread-local stack.
Objects are automatically labelled according to their local variable name whenever a \enquote{with}-scope is left.

Calling the \enquote{\texttt{to\_svg}} method on the assembled neuron object (or simply evaluating the object in a Jupyter cell) generates an annotated \enquote{ball-and-stick} representation of the neuron (similar to those depicted in~\Cref{fig:nlif}) using \emph{GraphViz} \citep{ellson2004graphviz}.

\subsection{Simulating $n$-LIF Neurons}

\begin{figure}
	\centering
	\includegraphics{media/chapters/ZD_software/libnlif_simulation_example.pdf}
	\label{fig:libnlif_simulation_example}
	\caption[Output of the \emph{libnlif} two-compartment LIF simulation code example]{Output of the \emph{libnlif} two-compartment \LIF simulation code example. Output spikes are available as individual spike times at sub-sample resolution or as a discretised sum of Dirac deltas.}
\end{figure}

The assembled $n$-LIF object holds the system matrices \mnAp, \mnBp, \vnap, \vnbp, \mnL (\Cref{sec:nlif_description}).
These matrices can be accessed through the \texttt{A}, \texttt{B}, \texttt{a\_const}, \texttt{b\_const} and \texttt{L} properties and are used by \emph{libnlif} to simulate the dynamics of the neuron.
Specifically, the neural dynamics can be simulated using the \texttt{nlif.Simulator} object and one of the available \texttt{simulate} methods:
\begin{pythoncode}
dt, ss, T = 1e-4, 10, 1.0     # Simulation time-step, sub-sampling and end-time
ts = np.arange(0, T, ss * dt) # Sample points
gEs = np.linspace(0.0, 200e-9, len(ts)) # Sampled input conductances
with nlif.Simulator(two_comp_lif_assm, dt=dt, ss=ss,
                    record_voltages=True, record_spike_times=True) as sim:
	res = sim.simulate({      # Also: simulate_poisson, simulate_filtered
		gE: gEs,              # Sampled input
		gI: 10e-9})           # Constant input
plt.plot(ts, res.v)           # Discretised output pulses are stored in res.out
\end{pythoncode}
The output of this code is depicted in \Cref{fig:libnlif_simulation_example}.
Other variants of the \enquote{simulate} method include \texttt{simulate\_poisson}, which generates artificial Poisson-distributed spike noise \citep[e.g.,][Section~1.4]{abbott2001theoretical} and treats the given inputs as expectation values, as well as \texttt{simulate\_filtered}, which filters the input using a first-order low-pass.

\subsubsection{Automatic simulator generation and compilation}
Instantiating the \texttt{Simulator} class transparently generates a dynamically linked library that implements a specialised simulator.
The library is based on a templated \citep[Chapter~23]{stroustrup2013programming} C++ simulator (cf.~\texttt{simulator.hpp});
substituting in the specific system matrices and simulator options and compile-time maximises run-time performance.
The code is compiled using a custom C++ build system (see \texttt{cmodule.py}) that calls out to the GNU Compiler Collection\footnote{See \url{https://gcc.gnu.org/} for more information.} and caches the generated object files according to their input file hashes.
Reentrancy of the build system is ensured by employing an idempotency mechanism based on temporary target files and the atomic \enquote{rename} filesystem operation \citep[pp.~1816-1820]{2018ieee}.
The compiled library is loaded via the Python \texttt{ctypes} foreign function interface; the C++ code directly operates on the memory regions that back the Numpy arrays.


Internally, the simulator uses the Eigen linear algebra library \citep{eigenweb}.
The dynamical system is integrated under a zero-order hold assumption by solving for the system state at time $t + \Delta t$ according to the closed-form matrix exponential form \cref{eqn:nlif_dynamics}.

\subsubsection{Input sweeps}

\begin{figure}
	\centering
	\includegraphics{media/chapters/ZD_software/libnlif_rate_empirical_example.pdf}
	\caption[Output of the \emph{libnlif} \enquote{rate\_empirical} function for different noise parameters]{Output of the \emph{libnlif} two-compartment \LIF \texttt{rate\_empirical} rate example. The specified $\lambda$ corresponds to the rate of the Poisson spike source; the low-pass filter time constant $\tau$ is set to $\SI{5}{\milli\second}$.}
	\label{fig:libnlif_rate_empirical_example}
\end{figure}

The assembled neuron object provides convenience functions for sampling the multi-variate response curves $\mathscr{G}(\vec g)$ (eq.~\ref{eqn:def_response_curve_g}).
The following code simulates the neuron for \SI{100}{\second} and estimates the average firing rate by taking the inverse of the median inter-spike interval.
The optional \texttt{noise}, \texttt{rate}, and \texttt{tau} parameters specify Poisson-distributed spike-noise.
\begin{pythoncode}
gEs, gIs = np.linspace(0, 1e-6, 20), np.linspace(0, 1e-6, 20) 
gEss, gIss = np.meshgrid(gEs, gIs)        # Generate a dense sample grid
rates = two_comp_lif_assm.rate_empirical( # Simulate the neuron at each sample
	{gE: gEss, gI: gIss}, T=100.0, noise=True, rate=10000, tau=5e-3)
\end{pythoncode}
Evaluation of the sample points is distributed across all available processor cores.
The output of the above code-snippet (with a $100 \times 100$ grid) is visualised in \Cref{fig:libnlif_rate_empirical_example} for different rates $\lambda$.

\subsection{Predicting Somatic Currents and Solving For Parameters and Weights}

\begin{figure}
	\centering
	\includegraphics{media/chapters/ZD_software/libnlif_parameter_estimation.pdf}%
	{\phantomsubcaption\label{fig:libnlif_parameter_estimation_a}}%
	{\phantomsubcaption\label{fig:libnlif_parameter_estimation_b}}%
	{\phantomsubcaption\label{fig:libnlif_parameter_estimation_c}}%
	\caption[Two-compartment LIF parameter optimisation using \emph{libnlif}]{Two-compartment \LIF parameter optimisation using \emph{libnlif}.
	Filled contours are the empirical error measurements, dashed white lines the predicted somatic currents.}
	\label{fig:libnlif_parameter_estimation}
\end{figure}

The assembled neuron object can also predict somatic currents according to \cref{eqn:h_model}.
To this end, we must first compute the reduced system matrices \mrAp, \mrBp, \vrap, \vrbp, \mrL, \vrc:
\begin{pythoncode}
sys = two_comp_lif_assm.reduced_system(v_som=None) # v_som = 0.5*(v_reset+v_th)
\end{pythoncode}
%Here, \texttt{v\_som=None} implies that the average somatic potential \vSom is computed as the mean between the reset and threshold potential.
The reduced system matrices are stored in the \texttt{A}, \texttt{B}, \texttt{a\_const}, \texttt{b\_const}, \texttt{L} and \texttt{c} properties of the \texttt{sys} object.
Currents can be predicted as follows:
\begin{pythoncode}
i_som_pred = two_comp_lif_assm.i_som({gE: gEss, gI: gIss}, reduced_system=sys)
\end{pythoncode}

\subsubsection{Estimating model parameters}
As depicted in \Cref{fig:libnlif_parameter_estimation_a}, \texttt{i\_som\_pred} is not very accurate without further calibration.
The \texttt{nlif.parameter\_optimisation} package implements the optimisation methods discussed in \Cref{sec:nlif_opt_parameters}, including our 
our soft trust-region based \SQP:
\begin{pythoncode}
gs_train = two_comp_lif_assm.canonicalise_input({gE: gEss, gI: gIss})
Js_train = two_comp_lif_assm.lif_rate_inv(rates) # Invert the response curve
valid = rates > 12.5                             # Discard subthreshold samples
sys = sys.condition()                            # Condition the reduced system
sys_opt, errs_train = nlif.parameter_optimisation.optimise_trust_region(
	sys, gs_train=gs_train[valid], Js_train=Js_train[valid], N_epochs=10)
\end{pythoncode}
The optimised system is stored in \texttt{sys\_opt} and can be passed to the above \texttt{i\_som} method to obtain improved current predictions (\Cref{fig:libnlif_parameter_estimation_b,fig:libnlif_parameter_estimation_c}).
Just like the \nlif simulator, the trust-region optimiser relies on a dynamically compiled C++ library that in turn makes use of the Eigen and OSQP libraries (\cite{stellato2020osqp}; see \texttt{nlif\_solver\_parameters.cpp}).

\subsubsection{Solving for synaptic weights}

\begin{figure}
	\includegraphics{media/chapters/ZD_software/libnlif_weight_solver.pdf}
	\caption[Nonnegative multiplication in two-compartment LIF neurons using \emph{libnlif}]{Nonnegative multiplication in two-compartment \LIF neurons using \emph{libnlif}. Coloured contours and dotted lines show the decoded currents; dashed lines are the target.}
\end{figure}

The synaptic weight solvers discussed in \Cref{sec:nlif_opt_weights} are provided in the \texttt{nlif.weight\_optimisation} package.
For example, to find synaptic weights that approximate non-negative multiplication we can use the following code:
\begin{pythoncode}
# Sample the represented space
xs, ys = np.linspace(-1, 1, 101), np.linspace(-1, 1, 101)
xss, yss = np.meshgrid(xs, ys)
Xs = np.array((xss.flatten(), yss.flatten())).T

# Obtain the target current function and pre-activities
Js = 0.5 * (1.0 + Xs[:, 0]) * (1.0 + Xs[:, 1]) * 1e-9
As = [...] # (N_smpls x m_pre); use the \NEF tuning curve equations here

# Perform the actual training
idcs = np.random.randint(0, Xs.shape[0], 256)  # Training samples
W_mask = np.ones((2, As.shape[1]), dtype=bool) # All-to-all connectivity
W, errs_train = nlif.weight_optimisation.optimise_trust_region(sys_opt,
    As_train=As[idcs], Js_train=Js[idcs], W_mask=W_mask, N_epochs=10)

# Compute the decoded currents
Js_dec = two_comp_lif_assm.i_som(As @ W.T, reduced_system=sys_opt)
\end{pythoncode}
Again, the optimiser is implemented in a C++ library (see \texttt{nlif\_solver\_weights.cpp}).
Since the generated quadratic programs tend to be moderately large (on the order of thousands of variables, depending on the number of pre-neurons and samples), the code makes extensive use of sparse matrices.
To prevent heap reallocations while populating the matrices, the code pre-calculates the number of non-zero entries per column and sequentially writes the matrix coefficients to memory.
The low-level \texttt{nlif.Solver} class wraps the C++ code more directly and supports concurrent optimisation across multiple post-neurons.
