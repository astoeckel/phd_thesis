\usepackage{amsmath}
\usepackage{amsthm}
\usepackage{amssymb}
\usepackage{graphicx}
\usepackage{csquotes}
\usepackage{mathrsfs} % mathscr
\usepackage{mathtools}
\usepackage{booktabs}
\usepackage{xspace}
\usepackage{longtable}
\usepackage{multirow}
\usepackage{rotating}
\usepackage{nicematrix}
\usepackage{algpseudocode}
\usepackage[chapter]{algorithm}

\usepackage{marvosym} % Lightning symbol

% Restateable theorems
\usepackage{thmtools}
\usepackage{thm-restate}


%\usepackage{imakeidx}

\usepackage[toc,acronyms,symbols,nopostdot]{glossaries} % Glossary.
\usepackage{glossaries-extra}

\usepackage[labelfont={small,bf,sf},textfont={small,sf}]{caption}
\DeclareCaptionStyle{ruled}{labelfont={small,bf,sf},textfont={small,sf}}
\usepackage{subcaption}

\usepackage[a-1b]{pdfx}

% Setup Bibtex

\usepackage[
	style=authoryear,
	uniquename=false,
	mincitenames=1,
	maxcitenames=1,
	citetracker=true,
	natbib=true,
	sorting=nyt,
	hyperref=true,
	backref=false]{biblatex}
\setlength{\bibnamesep}{0.5em}


% See https://tex.stackexchange.com/a/478787
\makeatletter
\AtEveryCitekey{%
  \ifciteseen
    {}
    {\defcounter{maxnames}{3}}%
}
\makeatother

\usepackage{siunitx}

%\DefineBibliographyStrings{english}{%
%  backrefpage = {cited on page~},
%  backrefpages = {cited on pages~},
%}

\DeclareFieldFormat{citehyperref}{%
  \DeclareFieldAlias{bibhyperref}{noformat}% Avoid nested links
  \bibhyperref{#1}}

\DeclareFieldFormat{textcitehyperref}{%
  \DeclareFieldAlias{bibhyperref}{noformat}% Avoid nested links
  \bibhyperref{%
    #1%
    \ifbool{cbx:parens}
      {\bibcloseparen\global\boolfalse{cbx:parens}}
      {}}}


\definecolor{linkBlue}{HTML}{003399}
\definecolor{linkGreen}{HTML}{2E8B57} % SeaGreen
\definecolor{linkPink}{HTML}{FF1493} % SeaGreen

\usepackage{hyperref} % with basic options
\hypersetup{
    pdfborder={0 0 0},
    plainpages=false,
    unicode=true,
%    colorlinks=false,
    colorlinks=true,
    allcolors=linkPink,
    citecolor=linkGreen
}
\usepackage{cleveref}
\renewcommand\thesubfigure{\Alph{subfigure}}

\declaretheorem[numberwithin=chapter]{theorem}
\declaretheorem[name=Claim,sibling=theorem]{claim}
\declaretheorem[name=Conjecture,sibling=theorem]{conjecture}
\declaretheorem[name=Proposition,sibling=theorem]{proposition}
\declaretheorem[name=Lemma,sibling=theorem]{lemma}

\declaretheorem[name=Definition,numberwithin=chapter]{definition}
